\documentclass[10pt,twocolumn]{article}

% use the oxycomps style file
\usepackage{oxycomps}

% usage: \fixme[comments describing issue]{text to be fixed}
% define \fixme as not doing anything special
\newcommand{\fixme}[2][]{#2}
% overwrite it so it shows up as red
\renewcommand{\fixme}[2][]{\textcolor{red}{#2}}
% overwrite it again so related text shows as footnotes
%\renewcommand{\fixme}[2][]{\textcolor{red}{#2\footnote{#1}}}

% read references.bib for the bibtex data
\bibliography{references}

% include metadata in the generated pdf file
\pdfinfo{
    /Title (Large Language Models for Video Game Dialogue: Advances in Dynamic Storytelling and Effects on Player Experience)
    /Author (Max Tulley)
}

% set the title and author information
\title{Large Language Models for Video Game Dialogue: The State of Dynamic Storytelling and Effects on Player Experience}
\author{Max Tulley}
\affiliation{Occidental College}
\email{tulley@oxy.edu}

\begin{document}

\maketitle

\section{Introduction and Problem Context}

\noindent The implementation of large language models (LLMs) in games has become an interesting topic of discussion among game designers and designers of mediated experiences more generally. Although this paper will focus mostly on video games, much of the information applies to other mediated experiences as well which implement AI controlled characters and agents. LLMs offer exciting new paths for many aspects of game design, including narrative design, dialogue generation, world generation, and coding. Perhaps the most obvious and direct use of LLMs in games is for generating dialogue. Many types of games make use of dialogue for storytelling, player direction, and NPC (Non-Playable Character) interaction. Large language models provide the opportunity for dynamic dialogue in which responses are semi-randomized, creating a more lifelike world and potentially increasing player engagement and immersion. A more immersive game world can lead to increased player emotional responses to the narrative, increased engagement with the game world, and a more fulfilling experience of playing the game. 

The research questions to be discussed in this paper are:
    \newline1) Can LLMs generate coherent and narratively consistent dialogue for video games? 
    \newline2) How does LLM generated dialogue affect player experience?
    \newline3) What are the current technical limitations in implementing LLMs for game dialogue?
    \newline4) What is the current state of dynamic storytelling through LLM in the academic literature?


% Citations for references page to show up: \cite{akoury_framework_2023}\cite{christiansen_exploring_2024}\cite{cox_conversational_2024}\cite{csepregi_effect_nodate}\cite{gallotta_large_2024}\cite{huang_generating_2024}\cite{kostilainen_next_2024}\cite{kostilainen_next_2024}\cite{marincioni_effect_2024}\cite{nagarkar_improving_2024}\cite{nananukul_what_2024}

\section{Technical Background}
It's important to understand the common implementations of dialogue systems in games in order to understand why LLMs offer such an exciting alternative. Most games use fixed/scripted dialogue systems. Game developers will write scripts for each character and conversation. The player can often select from a few dialogue options that will influence the course of the conversation. This offers limited conversational flexibility and can be predictable and consequently uninteresting. LLMs offer the ability for dialogue to be generated in real-time and allow for the player to input anything they want. This simulates real conversation much more accurately and would add a more lifelike quality to all conversations. 

\section{Literature Review}
This section will review recent academic literature which relates to LLMs in video games. This literature review does not follow a systematic approach, however, the process will be described here. The articles discussed were found using the following search query in multiple online databases: "("Video games" OR "games") AND ("LLM" OR "Large Language Model")". The articles were then selected manually, and any articles that did not discuss real-time dialogue generation were disregarded. 

\subsection{Use Cases}
This section will explain the common use cases for LLMs in video games discussed in the academic literature. These include real-time dialogue generation, narrative generation, procedural game world generation, and game programming.

\subsubsection{Dialogue Generation}
This topic is the most relevant for this paper since this project will focus on creating dialogue with LLMs. There are two options for LLM generated dialogue. The LLM can generate dialogue during the game development process which would then be implemented into a fixed dialogue system. The other option is to use the LLM for real-time dialogue that is generated while the game is played. The second option is much more revolutionary than the first as it allows for a completely new way to control game dialogue. Because of this, most recent research in this field examines the real-time generation of LLM dialogue.

\subsubsection{Narrative Generation}
This topic covers both real-time dynamic narrative creation and narrative creation during the game development process.


\subsection{Types of Games}
This section will evaluate the types of games that previous articles have written about. Some common themes among these papers are the focus on role-playing games, virtual/extended reality, and pregenerated narrative games.

\subsubsection{Role-Playing Games}
One of the most frequently mentioned genres of video games in video game LLM papers are role-playing games (RPGs). There is some debate over the definition of role-playing game, like whether the player-character must be a blank slate, but this paper will use the following definition. Role-playing games are games in which the player chooses player attributes such as skills, stats, appearance, and personality. The player's attributes and choices will affect the gameplay and narrative of the game. This genre of video game would benefit massively from the successful integration of LLMs since much of the gameplay of RPGs is dialogue. CRPGs (Computer RPGs) originate from TTRPGs (Table-top RPGs) like Dungeons and Dragons in which the Dungeon Master acts as the NPCs. This allows for incredible emergent gameplay and dynamic NPC-player interaction, since the NPC dialogue is improvised and acted on by a person. CRPGs lack this dynamic and unpredictable NPC interaction. LLMs could potentially add an AI dungeon master to CRPGS and simulate the randomness and unpredictability of real conversation. CRPGs mostly use scripted dialogue which is less interactive than real conversation, since there are only a few dialogue options for the player to choose and a few responses that the NPC can give. Comparing this with an LLM operated NPC where the player can say anything and the NPC can give real-time responses. Clearly an LLM NPC would be much more interactive in a game and simulate real life conversation more accurately. This would make the game world more believable and give the player the feeling of having more control over their character and their conversations with NPCs.

\subsubsection{Extended Reality}
Extended reality is an umbrella term which covers virtual reality, augmented reality, and mixed reality. It is a topic which comes up frequently in the academic research of LLMs in games and for good reason. The goal of implementing LLM-NPCs is to create a more immersive and lifelike world. This complements extended reality as it too seeks to increase player immersion in the game/media world. Many studies use text-to-speech and speech-to-text for their dialogue systems. This simulates reality better than typing dialogue into a textbox or choosing dialogue from dialogue options. Text-to-speech offers the ability for LLM dialogue to be spoken by an AI voice, so that the NPC can talk to the player. This paper suggests that using text-to-speech actually harms player immersion since AI-voices often fall into the uncanny valley and players can immediately identify an AI generated voice line. Additionally, many studies have identified the delay of LLM response generation to be a technical problem with LLM integration in games. This causes a problem in extended reality games since players are expecting NPCs to act more life-like. In non-extended reality games, say for example a game with text boxes for dialogue, the delay is not as much of a problem since the player will not be expecting lifelike speech behavior from NPCs. The problem of LLM repsonse time can also be mitigated with rolling text, so that the response is given to the player as the LLM is generating it, much like Chat-GPT 4 does.

\subsubsection{Narrative Generation}

\section{Dialogue Coherence and Consistency}

The first question that this paper aims to answer is: "Can LLMs generate coherent and narratively consistent dialogue for video games?" Short answer, yes. There are already games that use LLMs for dialogue and other game-play.

\section{Player Experience}

\subsection{Evaluation Methods}

\subsection{Believability of LLM-powered NPCs}

\subsection{Immersion}

\subsection{NPC Response Time}

\section{Technical Challenges}

\section{Ethical Considerations}

\section{The Future of LLMs in Games}

\section{Conclusion}

\printbibliography

\end{document}
